\documentclass{ees-repcn}
\makeindex
\begin{document}
\title{XXX教程}                % 标题。上限 17 个汉字宽度,不可以换行。
\subtitle{}             % 可选副标题。上限 20 个汉字,不可以换行。
                        % 正副标题一共上限 30 个汉字。
%\date{}                % 日期。
                        % 括弧内留空:封面不出现日期。
                        % 注释掉此行:日期为编译当天。

\author{}               % 责任作者姓名。限一人。多作者时填在“修订记录”中。
\docattr{
         docid={EES-XX-xxx},        % 文档编号
%         relatedid={},             % 关联文档编号,可填多个,逗号隔开。
         email={},                  % 责任作者邮箱地址。
         classification={公开},    % 密级。
         type={},                   % 文档分类。
         status={}}                 % 文档状态。
\maketitle

\begin{abstract}         %  文档摘要
% 本文介绍了。。。
\end{abstract}

\begin{revisions} % 修订记录,格式如下:
%  0.1 & 2013.12.25 & 赵军 & draft \\
%  0.2 & 2014.01.04 & 沈忱 & translate .tex to .cls \\
\end{revisions}

\frontmatter            % 封面页
\tableofcontents        % 目录页
%\clearpage\listoffigures\listoftables\lstlistoflistings  % 根据需要,可以有插图目录、表的目录和代码目录

\mainmatter
正文从这里开始。
\LaTeX\cite{oetiker1995not} is great!
% \input{introduction} \input{problem} % 建议章节用独立的 tex 文件。


\appendix

\bibliographystyle{ieeetr}
\bibliography{main}

\printindex{}
\backmatter
\end{document}
